\begin{multicols}{2}
    \en{
        \CvSection{Work Experiences}{
            \CvSectionEntry{R\&D Software Engineer}{
                \textbf{Date:} 07/2018 - 08/2020 (2 year 1 months) \\
                \textbf{Company:} Cantek Group - \textbf{Place:} Antalya \\
                \textbf{Goal}: Development of Industrial Control Systems (ICS)
                and Early Warning Systems (EWS) with local device or
                cloud server based web applications for industrial
                storage and vertical plant factories. \\
                \textbf{Job Descirption:} \\
                ICS systems are generally managed autonomously by
                controller devices consisting of PLC or electronic cards
                developed with embedded software. By communicating
                with the controller devices, the necessary data is sent to
                the monitoring system designed with micro service
                architecture in cloud servers. Designing software that
                offers analytical solutions over large data sets created in
                the cloud environment, developing it with software and
                embedded software teams. \\
                \textbf{-} Using Hybrid Database Architectures with combining
                PostgreSQL + Cassandra technologies \\
                \textbf{-} Developing Java Microservices with Spring Boot
                (mostly Spring Framework Core, Spring Security and
                Spring Cloud Modules) \\
                and deploying them as Docker containers on the server \\
                \textbf{-} Configure a system running on the Ubuntu 16.04 server
                and apply DevOps using CI/CD Softwares running inside separate Docker containers. \\
                \textbf{|} Managing source code via GitLab \\
                \textbf{|} Compiling Spring Boot applications with Maven via Jenkins \\
                \textbf{|} Serving JARs on the JFrog Artifactory via Jenkins  \\
                \textbf{Outcome}: A customizable system solution has been provided to solve the problems in the old system and to meet the requirements requested by the users from the old system.
            }

            \CvSectionEntry{Java Developer}{
                \textbf{Date:} 06/2016 - 06/2018 (2 year) \\
                \textbf{Company:} Akdeniz University BAUM - \textbf{Place:} Antalya \\
                \textbf{Goal:} Implementing a computer-based assessment system for educational assessment. \\
                \textbf{Job Description:} \\
                \textbf{-} Developing Java Web Apps (Spring MVC) with using Spring framework \\
                \textbf{-} Providing communication between web apps and database
                over RESTful Service \\
                \textbf{-} Compiling applications with Maven, sending them to the SVN environment,
                serving them on the Tomcat Server via Jenkins. \\
                \textbf{Outcome:} Use of information technology in assessment for educational assessment
            }

        }

        \colbreak

        \CvSection{Start-Up Projects}{
            \CvSectionEntry{Yazilim VIP (https://wwww.yazilim.vip)}{
                \textbf{Date:} 07/2018 - Present \\
                To keep this platform open to everyone who is
                competent and willing to share what we call an open
                source world. A platform aimed at presenting our
                competencies to the open source world without waiting
                for a response.
            }
        }

        \CvSection{Open-Source Projects}{
            Check my GitHub for my other Open-Source projects \\
            \textbf{https://github.com/maemresen}

            \CvSectionEntry{Spring VIP}{
                \textbf{Link:} https://springvip.yazilim.vip/ \\
                A Java Spring Library that provides helpful generic and utility classes. Some of the features provided by the library are
                generic CRUD Service and Rest controllers. maescript
            }
            \CvSectionEntry{Maescript}{
                \textbf{Link:} https://yazilim-vip.github.io/maescript/ \\
                A very useful set of Linux commands
            }
            \CvSectionEntry{React HowTo}{
                \textbf{Link:} https://github.com/yazilim-vip/howto-react \\
                Reusable component library for ReactJS
            }
        }

        \CvSection{Side Projects}{
            \CvSectionEntry{IconConference2017 Website}{
                Custom-designed WordPress theme for the customer.
                The theme also includes a basic file upload mechanism. The theme was used for IconConference 2017 conference website.
            }
        }

        \CvSection{Senior Design Project}{
            \CvSectionEntry{
                Face Recognition based \\
                Class Attendance Tracking System
            }{
                The main motivaton of that project is creating a platform which makes use
                able to both taking and tracking attendance digitally without wasting time.
            }
        }


        \CvSection{Attended Organizations}{
            \textbf{GDG 2019 :} Participant \\
            \textbf{ICAIME 2019 :} Participant \\
            \textbf{Linux Seminar: }Undergraduate Seminar \\
            \textbf{3. Antalya Bilim Şenliği :} Participant \\
            \textbf{UBMK 17 :} Crew \\
        }
    }

    \tr{
        \CvSection{Çalışma Deneyimleri}{
            \CvSectionEntry{Ar-Ge Yazılım Mühendisi}{
                \textbf{Tarih:} 07/2018 - 08/2020 (2 yıl 1 ay) \\
                \textbf{Şirket:} Cantek Group - \textbf{Yer:} Antalya \\
                \textbf{Hedef}: Endüstriyel depolama ve dikey fabrika fabrikaları için lokal 
                cihaz veya bulut sunucu tabanlı web uygulamaları ile Endüstriyel Kontrol Sistemleri (EKS)
                ve Erken Uyarı Sistemlerinin (EUS) Geliştirilmesi. \\
                \textbf{İş Tanımı:} \\
                EKS sistemleri, genellikle gömülü yazılım ile geliştirilen PLC veya elektronik kartlardan oluşan 
                denetleyici cihazlar tarafından özerk olarak yönetilir. Denetleyici cihazlar ile haberleşerek gerekli 
                veriler bulut sunucularda mikro servis mimarisi ile tasarlanmış izleme sistemine gönderilir. 
                Bulut ortamında oluşturulan büyük veri kümeleri üzerinden analitik çözümler sunan yazılımlar 
                tasarlamak, yazılım ve gömülü yazılım ekipleri ile geliştirme yapmak. \\
                \textbf{-} İlişkisel ve NoSQL veritabanlarının kullanılması (PostgreSQL + Cassandra) \\
                \textbf{-} Spring Boot ile Java Microservislerinin geliştirilmesi
                (Spring Framework Core, Spring Security ve Spring Cloud Modülleri) \\
                and deploying them as Docker containers on the server \\
                \textbf{-} Ubuntu 16.04 Sunucu yönetimi
                \textbf{-} Git ve GitLab ile versiyon kontrol sistemi \\
                \textbf{-} Jenkins üzerinden Spring Boot uygulamalarının Maven ile derlenmesi \\
                \textbf{-} Build edilen JAR'ların  JFrog Artifactory üzerinde deplolanması  \\
                \textbf{Çıktı}: Eski sistemdeki sorunların giderilmesi ve eski sistemden 
                kullanıcıların talep ettiği ihtiyaçların karşılanması için özelleştirilebilir 
                bir sistem çözümü sunulmuştur.
            }

            \CvSectionEntry{Java Geliştirici}{
                \textbf{Tarih:} 06/2016 - 06/2018 (2 yıl) \\
                \textbf{Şirket:} Akdeniz University BAUM - \textbf{Yer:} Antalya \\
                \textbf{Hedef:} Eğitimsel değerlendirme için bilgisayar tabanlı bir değerlendirme sistemi uygulamak. \\
                \textbf{İş Tanımı:} \\
                \textbf{-} Sprig MVC ile Java Web uygulamaları geliştirilmesi \\
                \textbf{-} RESTful Servisler üzerinden web uygulamaları ile veritabanı arasında iletişim sağlanması \\
                \textbf{-} Compiling applications with Maven, sending them to the SVN environment,
                serving them on the Tomcat Server via Jenkins. \\
                \textbf{Çıktı:} Eğitimsel değerlendirme için bilgi teknolojilerinin kullanımı
            }
        }

        \colbreak

        \CvSection{Start-Up Projeler}{
            \CvSectionEntry{Yazilim VIP (https://wwww.yazilim.vip)}{
                \textbf{Tarih:} 07/2018 - Şu An \\
                Bu platformu yetkin ve açık kaynaklı dünya dediğimiz şeyi paylaşmaya istekli herkese açık tutmak. 
                Yetkinliklerimizi yanıt beklemeden açık kaynak dünyasına sunmayı amaçlayan bir platform.
            }
        }

        \CvSection{Açık Kaynak Kodlu Projeler}{
            Diğer Açık Kaynak projelerim için GitHub'ımı kontrol edebilirsiniz \\
            \textbf{https://github.com/maemresen}

            \CvSectionEntry{Spring VIP}{
                \textbf{Link:} https://springvip.yazilim.vip/ \\
                Yararlı genel ve yardımcı sınıflar sağlayan bir Java Spring kütüphanesi. 
                Kütüphane tarafından sağlanan özelliklerden bazıları genel CRUD 
                Servisi ve Rest Controller'dır.
            }
            \CvSectionEntry{Maescript}{
                \textbf{Link:} https://yazilim-vip.github.io/maescript/ \\
                Kullanışlı Linux komutları
            }
            \CvSectionEntry{React HowTo}{
                \textbf{Link:} https://github.com/yazilim-vip/howto-react \\
                ReactJS için yeniden kullanılabilir component kitaplığı
            }
        }

        \CvSection{Yan Projeler}{
            \CvSectionEntry{IconConference2017 Websayfası}{
                Müşteri için özel olarak tasarlanmış WordPress teması.
                Tema ayrıca basit bir dosya yükleme mekanizması içerir. 
                Tema, IconConference 2017 konferans web sitesi için kullanılmıştır.
            }
        }

        \CvSection{Bitirme Projesi}{
            \CvSectionEntry{
                Yüz Tanımı tanımlı Sınıf Yoklama Takip Sistemi
            }{
                Bu projenin temel motivasyonu, zaman kaybetmeden katılımı dijital olarak hem almayı
                hem de takip etmeyi sağlayan bir platform oluşturmaktır.
            }
        }

        \CvSection{Katılınılan Etkinlikler}{
            \textbf{GDG 2019 :} Katılımcı \\
            \textbf{ICAIME 2019 :} Katılımcı \\
            \textbf{Linux Seminar: }Lisans Semineri \\
            \textbf{3. Antalya Bilim Şenliği :} Katılımcı \\
            \textbf{UBMK 17 :} Ekip \\
        }
    }

\end{multicols}